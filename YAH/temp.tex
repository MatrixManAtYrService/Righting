

\documentclass[12pt]{article}
\usepackage[letterpaper]{geometry}
\usepackage{times}
\geometry{top=1.0in, bottom=1.0in, left=1.0in, right=1.0in}

%
%Doublespacing
%
\usepackage{setspace}
\doublespacing


\usepackage{amsmath}
\usepackage{amssymb}
\usepackage{lplfitch}
\usepackage{caption}
\usepackage{mwe}
\usepackage{subcaption}
\usepackage{endnotes}

\usepackage{fancyhdr}
\pagestyle{fancy}
\setlength{\headheight}{110pt}
\lhead{}
\chead{}
\rhead{Rixman \thepage}
\lfoot{}
\cfoot{}
\rfoot{}
\renewcommand{\headrulewidth}{0pt}
\renewcommand{\footrulewidth}{0pt}


%Works cited environment
%(to start, use \begin{workscited...}, each entry preceded by \bibent)
% - from Ryan Alcock's MLA style file
%
\newcommand{\bibent}{\noindent \hangindent 40pt}
\newenvironment{workscited}{\newpage \begin{center} Works Cited \end{center}}{\newpage }


%
%Begin document
%
\begin{document}
\begin{flushleft}

%%%%First page name, class, etc
Matt Rixman\\
Welshon\\
Epistemology\\
May 15, 2017\\


%%%%Title
\begin{center}
    You are Here, a Logical Metatheory of Knowability
\end{center}


%%%%Changes paragraph indentation to 0.5in
\setlength{\parindent}{0.5in}
%%%%Begin body of paper here

\texbf{Introduction}

In 2002 U.S. Secretary of Defence Donald Rumsfeld addressed the press with a meta-epistemological categorization.
He asserted that there are known-knowns, known-unknowns, and unknown-unknowns.
In the first section of this paper, I will argue for another category (potentially overlapping with the first two): known-unknowables.
If a proposition P is known to be unknowable, it means either that information that would justify P will never be available [1], or that our justification of P is doomed for some other reason.
It is the second kind of failure that I shall be concerned with.

It is easy to jump to the conclusion that any argument containing an unknowable proposition is useless, and ought to be avoided.
"Of what we cannot speak, we must pass over in silence," as Wittgenstein says.
I don't agree.
Even in the absence of accessable truth, a belief might be useful.
But I do think that a body of discourse that contains only knowable propositions should get special considerations.
Participation in such discourse requires that we avoid accidentally relying on such a proposition.
To this end, I will offer some modifications to traditional logic that aim to avoid reliance on unknowables.

Thirdly, I will argue that the relativism hypothesis presupposes a logical landscape that includes my modifications, and that the objectivist position presupposes an incompatible set of logic rules.
Because of this, the relativist fails to evaluate objectivist arguments in their native logical context, and visa-versa.
As examples of this, I will use either regime to analyze some arguments against relativism that are given by Welshon and Boghossian.

Consider a pair of relativists that are debating the mertis of $T_p$, a theory where p holds, versus $T_{\neg p}$, where p does not hold.
Elsewhere, there is a pair of absolutists that are arguing about whether or not p is true.
In the last section I will argue that at least in some cases, the argumentative structure between relativists is \textit{so} similar to that of the absolutists that a translation function could be provided.
In these cases, the debate about relativism collapses to a disagreement regarding style--the issue can be solved by simply using the translation function to render your opponents argument in your favored system.
In cases where this equivalence does not hold, it is hoped that the argumentative structure exposed by my amendments to logic will shed some light on the nature of the rift.

\section{Unknowable Propositions}
foo

\begin{figure}[h]
    \centering
    \begin{subfigure}{\linewidth}
        \centering
        \begin{tabular}{l|c}
            \hline
            $\alpha$ & \textit{the axioms of number theory}\\
            S & Numbers have successors.\\
            E & Numbers can be even.\\
            H & The computer will halt.\\
            \hline
        \end{tabular}
        \label{fig:sub1}
    \end{subfigure}\\
    \par\bigskip
    \begin{subfigure}{.5\linewidth}
        \centering
        \fitchctx
        {
            \pline{\bigstar}\\
            \subproof
            {
                \pline{\mathnormal{\alpha}}
            }
            {
                \pline{S} \\
                \pline{E}
            }
            \pline{\lnot H}
        }
        \caption{A mathematician's view}
        \label{fig:sub2}
    \end{subfigure}%
    \begin{subfigure}{.5\linewidth}
        \centering
        \fitchctx
        {
            \subproof
            {
                \pline{S} \\
                \pline{E}
            }
            {
                \pline{\bigstar}
            }
        }
        \caption{the computer's view}
        \label{fig:sub3}
    \end{subfigure}
    \label{fig:test}
\end{figure}


\begin{figure}[h]
\begin{subfigure}{.5\textwidth}
    \fitchctx
    {
        \pline{\bigstar}\\
        \subproof
        {
            \pline{\mathnormal{R}}
        }
        {
            \subproof
            {
                \pline{\mathnormal{x}}
            }
            {
                \pline{}
            }
            \subproof
            {
                \pline{\lnot\mathnormal{x}}
            }
            {
                \pline{}
            }
        }
        \subproof
        {
            \pline{\neg \mathnormal{R}}
        }
        {
            \subproof
            {
                \pline{\mathnormal{x}}
            }
            {
                \pline{}
            }
        }
    }
    \caption{Relativism as relatively true}
\end{subfigure}%
\begin{subfigure}{.5\textwidth}
    \fitchprf
    {
        \pline{\mathnormal{R}}
    }
    {
        \pline{\bigstar}\\
        \subproof
        {
            \pline{\mathnormal{x}}
        }
        {
            \subproof
            {
                \pline{\lnot\mathnormal{x}}
            }
            {
                \pline{\lfalse}
            }
        }
    }
    \caption{Relativism as the only absolute}
\end{subfigure}
\end{figure}


\begin{figure}[h]
    \begin{tabular}{l|c|l}
        \hline
        Proposition & Symbol & Expression \\ \hline
        Relativism  & $R$ & $\forall{x}((\exists{\gamma})(x \in \gamma) \lif (\exists{\delta})({\lnot}x \in \delta))$ \\
        Antirelativism & $\neg R$ & $\exists{x}((\exists{\gamma})(x \in \gamma) \land (\forall{\delta})(({\neg}x \notin \delta))$ \\
        Absolutism & $A$ & $\exists{x}(\forall{\gamma})(x \in \gamma)$ \\
        Antiabsolutism & $\neg A$ & $\forall{x}(\exists{\gamma})({\lnot}x \in \gamma)$ \\
    \end{tabular}
\end{figure}

\begin{center}
    Notes
\end{center}
\setlength{\parindent}{0.5in}

1. A note

2. Another note

\begin{workscited}
    \bibent
    Some guy.  
\end{workscited}

\end{flushleft}
\end{document}
