\documentclass{article}
\usepackage[utf8]{inputenc}
\usepackage{mathtools}
\usepackage{amssymb}
\usepackage{enumerate}
\usepackage{lplfitch}
\usepackage[parfill]{parskip}

\newcommand{\indentitem}{\setlength\itemindent{25pt}}

\title{Point Set Topology as a Metatheory}
\author{Matt Rixman }
\date{March 2016}

\begin{document}

\maketitle

\section{Theorem Space}

Let \(T\) be a set, the elements of which are propositions.  
Each element of \(T\) is either an axiom or a theorem.
Also for any proposition \(p \in T\), we also have \(\neg p \in T\).
Finally, \(T\) contains at least two elements: an axiom (\(a\)) and its negation(\(\neg a\)).

Consider the following collection of subsets of \(T\):

\[h = \left\lbrace\begin{minipage}{8cm}
        \(S \subset T : \text{for each } s \in S \text{ one of the following holds}\)
        \begin{enumerate}[(i)]
            \indentitem \item \(s \text{ is itself an axiom}\)
            \indentitem \item \(s \text{ is provable from other axioms in } S\)
    \end{enumerate}
\end{minipage}\right\rbrace\]

We claim that \(T_h\) is a topological space.  For proof, note that:
\begin{enumerate}
    \item The above condition holds vacuously for \( \varnothing \), so \( \varnothing \in h\).
    \item Each \(p \in T\) is either an axiom, or can be proved from axioms in \(T\) in the following way:

        %http://mirrors.ibiblio.org/CTAN/macros/latex/contrib/lplfitch/lplfitch.pdf

\end{enumerate}



\end{document}

